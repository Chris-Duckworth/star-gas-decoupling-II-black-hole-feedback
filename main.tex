% mnras_template.tex 
%
% LaTeX template for creating an MNRAS paper
%
% v3.0 released 14 May 2015
% (version numbers match those of mnras.cls)
%
% Copyright (C) Royal Astronomical Society 2015
% Authors:
% Keith T. Smith (Royal Astronomical Society)

% Change log
%
% v3.0 May 2015
%    Renamed to match the new package name
%    Version number matches mnras.cls
%    A few minor tweaks to wording
% v1.0 September 2013
%    Beta testing only - never publicly released
%    First version: a simple (ish) template for creating an MNRAS paper

%%%%%%%%%%%%%%%%%%%%%%%%%%%%%%%%%%%%%%%%%%%%%%%%%%
% Basic setup. Most papers should leave these options alone.
\documentclass[fleqn,usenatbib]{mnras}

% MNRAS is set in Times font. If you don't have this installed (most LaTeX
% installations will be fine) or prefer the old Computer Modern fonts, comment
% out the following line
\usepackage{newtxtext,newtxmath}
% Depending on your LaTeX fonts installation, you might get better results with one of these:
%\usepackage{mathptmx}
%\usepackage{txfonts}

% Use vector fonts, so it zooms properly in on-screen viewing software
% Don't change these lines unless you know what you are doing
\usepackage[T1]{fontenc}
\usepackage{ae,aecompl}


%%%%% AUTHORS - PLACE YOUR OWN PACKAGES HERE %%%%%

% Only include extra packages if you really need them. Common packages are:
\usepackage{graphicx}	% Including figure files
\usepackage{amsmath}	% Advanced maths commands
\usepackage{amssymb}	% Extra maths symbols

%%%%%%%%%%%%%%%%%%%%%%%%%%%%%%%%%%%%%%%%%%%%%%%%%%

%%%%% AUTHORS - PLACE YOUR OWN COMMANDS HERE %%%%%

% Please keep new commands to a minimum, and use \newcommand not \def to avoid
% overwriting existing commands. Example:
%\newcommand{\pcm}{\,cm$^{-2}$}	% per cm-squared
\newcommand{\red}[1]{{\textcolor{red}{#1}}}
\newcommand{\green}[1]{{\textcolor{green}{#1}}}
\newcommand{\blue}[1]{{\textcolor{blue}{#1}}}

%%%%%%%%%%%%%%%%%%%%%%%%%%%%%%%%%%%%%%%%%%%%%%%%%%

%%%%%%%%%%%%%%%%%%% TITLE PAGE %%%%%%%%%%%%%%%%%%%

% Title of the paper, and the short title which is used in the headers.
% Keep the title short and informative.
\title[Decoupling the rotation of stars and gas - II]{Decoupling the rotation of stars and gas - II: the impact of black hole feedback on MaNGA kinematics in IllustrisTNG}

% The list of authors, and the short list which is used in the headers.
% If you need two or more lines of authors, add an extra line using \newauthor
\author[C. Duckworth et al.]
{Christopher Duckworth,$^{1,2}$\thanks{E-mail: cd201@st-andrews.ac.uk}
% Rita Tojeiro,$^{1}$
% Shy Genel,$^{2}$ 
% Tjitske Starkenburg,$^{2}$ \newauthor
% Melanie Habouzit$^{2,5}$
% Timothy Davis,$^{3}$
% Katarina Kraljic,$^{4}$
\\
% List of institutions
$^{1}$School of Physics and Astronomy, University of St Andrews, North Haugh, St Andrews, KY16 9SS, UK\\
$^{2}$Center for Computational Astrophysics, Flatiron Institute, 162 Fifth Avenue, New York, NY 10010, USA\\
}

% These dates will be filled out by the publisher
\date{Accepted XXX. Received YYY; in original form ZZZ}

% Enter the current year, for the copyright statements etc.
\pubyear{2019}

% Don't change these lines
\begin{document}
\label{firstpage}
\pagerange{\pageref{firstpage}--\pageref{lastpage}}
\maketitle

% Abstract of the paper
\begin{abstract}
We study the relationship between supermassive black hole (BH) feedback, BH luminosity and the kinematics of stars and gas for galaxies in IllustrisTNG. We use a sample of galaxies with mock MaNGA observations to identify kinematic misalignment at $z=0$ (difference in rotation of stars and gas), for which we follow the evolutionary history of BH activity and gas properties over the last 8 Gyrs. Misaligned low mass galaxies (quasar feedback) typically have boosted BH luminosity, BH growth and have had more energy injected into the gas over the last 8 Gyr in comparison to aligned galaxies. These properties likely lead to outflows and gas loss, in agreement with active low mass galaxies in observations. High mass quenched galaxies with misalignment (kinetic and quasar feedback) typically have similar BH luminosities, show no overall gas loss and have typically lower gas phase metallicity over the last 8 Gyrs in comparison to those aligned. We show that splitting on BH luminosity at $z=0$ produces statistically consistent distributions of kinematic misalignment at $z=0$, however, splitting on the maximum BH luminosity over the last 8 Gyrs does not. While BH activity and kinematic misalignment may correlate initially, galaxies remain misaligned on longer timescales making instantaneous correlation at $z=0$ difficult.
\end{abstract}

% Select between one and six entries from the list of approved keywords.
% Don't make up new ones.
\begin{keywords}
galaxies: kinematics and dynamics -- galaxies: active -- galaxies: evolution
\end{keywords}

%%%%%%%%%%%%%%%%%%%%%%%%%%%%%%%%%%%%%%%%%%%%%%%%%%

%%%%%%%%%%%%%%%%% BODY OF PAPER %%%%%%%%%%%%%%%%%%

\section{Introduction}
Tidal torque theory \citep[][]{hoyle1951, peebles1969, Doroshkevich1970} dictates that the angular momentum content of collapsing baryons are inherited from the surrounding dark matter halo. In the framework of a $\Lambda$ cold dark matter Universe, galaxies form from the cooling and condensation of the initial gas cloud within dark matter haloes \citep{fall1980, mo1998}. As stars form from the rotating gas, they inherit its dynamical characteristics often leading to coherent rotation between dark matter, gas and stars in both magnitude and direction. However, after turnaround there is good reason to believe that the rotation of dark matter, gas and stars may decouple from each other as galaxies evolve up-to $z=0$. 

Recent cosmological scale hydro-dynamical simulations have provided a clear insight into the relationship between the angular momentum of baryons and dark matter through cosmic time. A necessary component of realistic simulations is efficient feedback from both supermassive black holes (SMBH) and stars, required to, amongst other things, reproduce late type disks and solve the problem of catastrophic angular momentum loss \citep[e.g.][]{zavala2008, scannapieco2009}. Active galactic nuclei (AGN) and supernova explosions can also lead to dramatic redistribution of gas which regulate the angular momentum content of galaxies \citep[e.g.][]{genel2015, DeFelippis2017}. 
% Galactic winds carry both mass and angular momentum leading to potential boosts in angular momentum \citep[e.g.][]{genel2015, DeFelippis2017}. 

The exact impact of SMBHs on angular momentum is dependent on the mode of feedback. `Quasar' (radiative) mode feedback releases huge amounts of energy through radiation from the accretion disk leading to high luminosity AGN and dramatic gas outflows \citep[e.g.][]{cattaneo2009, rubin2014, cheung2016}. Alternatively `radio' (kinetic) mode is termed for lower luminosity AGN that host lower black hole accretion rates. In this instance energy is deposited into the surrounding gas via jets and winds which heat the gas and suppress star formation \citep[][]{binney1995, ciotti2001, heckman2014}.

The relationship between AGN and kinematics has been the focus of several recent studies using Integral Field Spectroscopy (IFS) data. In particular, a new class of galaxy termed `red geyser' has been identified which host AGN and exhibit high velocity outflows in the spatial distribution of ionized gas \citep[][]{cheung2016, roy2018}. These outflows are often linked to a distinctive offset in rotation direction between the stars and gas. Detection of ongoing outflows is, however, rare ($\sim5-10$\% of the quiescent population). \citet{penny2018} demonstrate the importance of AGN feedback in low mass quiescent galaxies. While the majority demonstrate no ionized gas present, quiescent galaxies with an active AGN that is actively suppressing star formation show a clear decoupling in the rotation of stars and gas. However, the relationship of gas kinematics to BH feedback is not clear for all galaxies \citep[see also:][]{koudmani2019}. \citet{ilha2019} find that typical rotation offsets between stars and gas for AGN defined galaxies are consistent with inactive controls. Termed kinematic misalignment, the decoupled rotation of stars and gas can also be a natural result of external processes \cite[e.g.][]{davis2011, barrera2015, vdvoort2015, jin2016, duckworth2019_halo}. Regardless of internal or external origin, kinematic misalignment is linked with both typical lower gas mass fractions and angular momentum \citep[e.g.][]{duckworth2019,starkenburg+19}. 

% Recent studies have shown a strong relationship between visual morphology and the likelihood of misalignment \citep[e.g. Duckworth et al. submitted,][]{chen2016, bryant2019}. Properties such as stellar mass are different when comparing aligned and misaligned galaxies of the same morphology and the difference (whether positive or negative) is morphologically dependent. Misaligned early-type galaxies are more massive, whereas misaligned late-types are less massive when compared to their aligned controls. This could be an indication that whether the misalignment is internally or externally driven is also morphologically dependent. \red{not relevant}

Misalignment appears to be ubiquitous in galaxies with ongoing feedback and outflows (that still have gas), however, the timescales of luminous AGN are typically much shorter than kinematic misalignment, making correlation at $z=0$ alone difficult. In this letter, we study the time evolution relationship between black hole feedback, black hole luminosity and kinematic misalignment in the cosmological scale hydrodynamical simulation of IllustrisTNG100 (hereafter referred to as TNG100). We use a sample of galaxies with mock MaNGA \citep[Mapping Galaxies at Apache Point;][]{bundy2015, blanton2017} observations at $z=0$ to emulate what we may expect to see in IFS observations. In Section \ref{sec:methods} we briefly describe the simulation and how we construct our sample. In Section \ref{sec:results} we present our results before concluding in Section \ref{sec:conclusion}.

\section{Methods} \label{sec:methods}
For this work, we use the fiducial run of TNG100 which follows the evolution of 2 x 1820$^3$ resolution elements within a periodic cube with box lengths of 110.7 Mpc (75 h$^{-1}$ Mpc). This corresponds to an average mass resolution of baryonic (dark matter) elements of 1.4 x 10$^6 M_{\odot}$ (7.5 x 10$^6 M_{\odot}$). The IllustrisTNG project \citep{marinacci18,naiman18,nelson18,pillepich18b,springel18} is a suite of magneto-hydrodynamic cosmological scale simulations incorporating an updated comprehensive model for galaxy formation physics \citep[][]{weinberger17,pillepich18a} and making use of the moving-mesh code \texttt{AREPO} \citep{springel10,pakmor11,pakmor13}. We make use of public data, as described in \citet{nelson2019}. Of particular relevance is the prescription of \citet{weinberger17} for BH feedback, which is modelled by two modes (quasar mode: high accretion, kinetic mode: low accretion). The transition between modes is dictated by both the BH mass and accretion rate onto the BH, so that as BHs grow over time, the kinetic mode is likely to become more dominant. 
% We note that in the kinetic model, the direction of injected momentum is random. 

To emulate what we may expect to see in IFS observations we construct a sample of TNG100 galaxies with mock MaNGA observations \citep[a complete description is given in][]{duckworth2019}. Here we briefly describe the sample construction. 
% Set to complete in 2020, the MaNGA survey is designed to investigate the internal structure of $\sim$10000 galaxies in the nearby Universe. By design, the complete sample is unbiased towards morphology, inclination and colour and provides a near flat distribution in stellar mass. 
For each MaNGA galaxy, we find an object in TNG100 with the closest match in stellar mass, $g-r$ colour and size. We create mock observations for each match by convolving the raw motions of all star particles/gas cells with noise and PSF typical of MaNGA. Each galaxy is `observed' up-to the typical MaNGA footprint (1.5-2.5 effective radii in a distinct hexagonal shape) to create mock velocity fields, from which we define the degree of misalignment between the rotation of stars and gas by fitting a position angle (PA). This is done using the \texttt{fit\_kinematic\_pa} routine \citep[see Appendix C of][]{krajnovic2006} so that $\Delta PA = |PA_{stellar} - PA_{gas}|$. We take objects with $\Delta$PA < 30$^{\circ}$ to be aligned, $\Delta$PA > 30$^{\circ}$ to be misaligned and $\Delta$PA > 150$^{\circ}$ to be counter-rotating. We only select galaxies that have distinct PAs (i.e. clear, coherent rotation) in both their stellar and gas velocity fields, leaving a total of $\sim$3600 galaxies used in this study. 

We estimate bolometric BH luminosities using the commonly adopted method in cosmological simulations: 
\begin{equation}
L_{bol} = \frac{\varepsilon_r}{1 - \varepsilon_r} \dot{M}_{BH} c^2
\end{equation}
where $\varepsilon_r$ is the radiative efficiency, c the light speed, and $M_{BH}$ the accretion rate onto the BH. We take $\varepsilon_r$ to be the commonly assumed value of 0.1 \citep[see discussion in][]{habouzit2019}.

\section{Results} \label{sec:results}
\begin{figure}
	\includegraphics[width=\linewidth]{overall_population/time_evolution_combo_all.pdf}
    \caption{Time evolution of (rows top to bottom) black hole luminosity ($log_{10}(L_{bol})$), gas fraction ($M_{gas}$/$M_{stel}$), cold gas fraction ($M_{gas}(T < 10^{5}K)$/$M_{stel}$), gas phase metallicity, gas angular momentum ($j_{gas}$), rate of energy injection from feedback and black hole mass for star forming (left) and quenched galaxies (right) identified at $z=0$. For the top 5 rows, the galaxies are split at $M_{stel} = 10^{10.2}M_{\odot}$ (corresponding to the transition from quasar to kinetic feedback) and also $\Delta$PA $< 30^{\circ}$ (solid) and $> 30^{\circ}$ (dashed) and  $> 150^{\circ}$ (dotted). Each line shows the mean for the population with the shaded region corresponding to the standard error. The bottom two rows (black hole energy injection and black hole mass) are residuals for the low mass galaxies only ($M_{stel} = 10^{10.2}M_{\odot}$) for $\Delta$PA$ > 30^{\circ}$ (dashed) and $\Delta$PA$ > 150^{\circ}$ (dotted) defined relative to $\Delta$PA$ < 30^{\circ}$. The horizontal line in energy injection panels represents the time at which 50\% of the energy over the last 8 Gyrs has been injected.
    }
    \label{fig:overall_pop}
\end{figure}

Each panel of Figure \ref{fig:overall_pop} shows the time evolution average of a property for all galaxies split by stellar mass at $M_{stel} = 10^{10.2}M_{\odot}$. This corresponds to the typical transition from quasar to kinetic feedback \citep[i.e. $M_{BH} \approx 10^{8}M_{\odot}$, see Fig 1 in][]{li2019}. We note that splitting in this way ensures that we isolate quasar feedback for the low mass sample, however, our high mass sample has been subject to both quasar and kinetic feedback over the last 8 Gyrs. Splitting on BH mass or enforcing stricter stellar mass cuts does not change any of our findings. We divide our sample by $\Delta$PA at $z=0$. For each sub-population, we find that the stellar mass distributions are consistent at $z=0$ for aligned and misaligned galaxies. We also split based on specific star formation rate at $z=0$ into star forming and quenched galaxies \cite[defined by distance from the star forming main-sequence in][]{pillepich2019}. 

In the top row we see distinct correlations between $\Delta$PA at $z=0$ and BH luminosity ($log_{10}(L_{bol})$) in the last 8 Gyr for low mass galaxies. This is apparent for both misaligned and counter-rotating galaxies. In contrast, we note that the correlation between misalignment and BH luminosity for the high mass bin is unclear. Given that low mass galaxies only exhibit quasar mode feedback, a possible interpretation is that this feedback mode is more likely to lead to misaligned gas rotation. Gas loss (all gas and cold phase to a similar degree; second and third rows) is a key feature for the low mass galaxies, potentially indicative of gas outflows from the feedback which also lowers the angular momentum content of the gas (fifth row - defined for all gas within a 3D radius of the IFS observational footprint). This is also matched by a boost in gas phase metallicity (fourth row - all gas cells) which could be explained by the radiative outflows preventing accretion of fresh material. 

In contrast, for the quenched high mass galaxies we see that the overall gas fraction for the misaligned remains constant with respect to those aligned, however the cold gas fraction relatively declines. In addition, we find that the gas phase metallicity is typically lower for those misaligned at $z=0$. While cold gas may be lost due to heating or BH accretion, this may indicate that accretion of pristine gas and gas rich minor mergers is more important for decoupling their rotation. This could be a natural result of quenched galaxies hosting smaller gas reservoirs, meaning that only a small amount of accretion (relatively to late-types) leads to misalignment. An alternate explanation could be that enriched gas is preferentially lost due to feedback or environment. Interestingly, the angular momentum content of the gas is disrupted to a similar degree as for the low mass galaxies.

For the low mass galaxies we note that the correlation (all panels), both in terms of timescale and magnitude for misaligned galaxies, is larger for those quenched at $z=0$. To understand this, in the bottom two panels we show the time evolution of the energy injection into the surrounding gas cells and black hole mass (see supplementary material for high mass equivalent of these panels). Here, we plot the average residual difference of misaligned (dashed) and counter-rotating (dotted) galaxies with respect to the aligned galaxies (grey dashed line at the origin). 

We find that the rate of energy injection is typically richer for counter-rotating galaxies, however a clear boost can be seen for all those that are misaligned. Interestingly, the lookback time of peak energy injection is earlier for quenched galaxies relative to the star forming (the solid horizontal line represents the time at which 50\% of the energy over the last 8 Gyrs has been injected). Star forming galaxies show far more recent feedback and BH luminosity, indicative that while the gas is in the process of being decoupled and blown out, it hasn't acted to fully suppress star formation yet. Conversely, looking at the quenched galaxies we can see that the feedback starts which in turn suppresses star formation leading to the quenched classification at $z=0$. For this reason selecting misaligned star forming and quenched galaxies at $z=0$ will naturally lead to different time correlations in BH activity. This can be visualized by the diagram in Figure \ref{fig:diagram}.

\begin{figure*}
	\includegraphics[width=\linewidth]{quasar_mode_feedback.pdf}
    \caption{Diagram of the relationship between BH accretion, radiative feedback, and kinematic misalignment in low mass galaxies. The diagram (left to right) shows the time evolution of radiative feedback which injects thermal energy into the cold gas reservoir of galaxy after a high accretion state. Below this are representations of stellar and gas velocity fields at each step showing what we may expect to see in an IFS observation if it was made at this time. $\Delta$PA refers to the difference between the global position angles (purple lines) for the stellar and gas velocity fields. High accretion onto the BH is directly related to boosted BH luminosity in our model. Over time radiative outflows act to suppress star formation leading to quiescence. Given that first misalignment ($\Delta$PA > 30$^{\circ}$) most commonly occurs around the high accretion period, we see that selecting different types of misaligned low mass galaxies at $z=0$ would result in time different correlation timescales of peak AGN luminosity. Star forming galaxies with misalignment due to feedback will have had more recent peak energy injection since the feedback hasn't had time to fully suppress star formation yet (as seen in the penultimate row of Figure \ref{fig:overall_pop}).}
    \label{fig:diagram}
\end{figure*}

We also show the residual time evolution of black hole mass with respect to the aligned galaxies. We find that BH growth correlates with the time scales of energy injection, indicating the close relationship between feedback and accretion for those misaligned. The causality between BH growth and feedback here is however not clear. While BH growth leads to increased feedback, the question remains whether the angular momentum is disrupted prior to feedback (potentially due to mergers leading to increased BH accretion and feedback) or if feedback drives the disruption of angular momentum and hence BH growth.

% \begin{figure}
% 	\includegraphics[width=\linewidth]{overall_population/LM_BH_residual_evo_Mstel10_2_twocol.pdf}
%     \caption{Time evolution of black hole energy injection (top) and black hole mass (bottom) for star forming (left) and quenched (right) galaxies identified at $z=0$. Shown are residuals for the low mass galaxies ($M_{stel} = 10^{10.2}M_{\odot}$) for $\Delta$PA$ > 30^{\circ}$ (dashed) and $\Delta$PA$ > 150^{\circ}$ (dotted) defined relative to $\Delta$PA$ < 30^{\circ}$. Each line shows the mean for the population with the shaded region corresponding to the standard error. The horizontal line in each of the top panels represents the time at which 50\% of the energy over the last 8 Gyrs has been injected.}
%     \label{fig:LM_BH}
% \end{figure}

To understand how kinematic misalignment may correlate with BH luminosity at $z=0$ alone, in the top panel of Figure \ref{fig:PAdist}, we show the distribution of $\Delta$PA for the top 20\% BH luminosity in our low mass sample ($M_{stel} < 10^{10.2}M_{\odot}$) in comparison with a control sample (all defined at $z=0$). The control is made by taking the closest unique match in stellar mass for each high BH luminosity galaxy from the remainder of our sample. We find the two distributions are statistically indistinct (Anderson-Darling statistic; -0.001 with a p-value of 0.348). In the bottom panel we show the same but instead we select the top 20\% in peak BH luminosity (for each galaxy) over the last 8 Gyrs. In this instance the AGN bright galaxies are distinctly more misaligned than the mass matched control (Anderson-Darling statistic; 13.793 with a p-value of 3e-5). 

This demonstrates that despite the inherent relationship between BH luminosity, feedback and gas kinematics, considering the overall distribution of $\Delta$PA split on instantaneous luminosity at $z=0$ does not necessitate that correlation is found. This matches that of observations in MaNGA (Figure 6 in \citet[][]{ilha2019}), who also find no correlation between active galaxies and a control. We emphasise that while feedback is a clear mechanism for driving misalignment, the decoupled rotation of gas can remain for several Gyr after initially becoming misaligned, meaning that a single time-step is unlikely to characterise the relationship for an ensemble of galaxies. 

\begin{figure}
	\includegraphics[width=\linewidth]{overall_population/PA_distribution_low_mass_z0_max_comparison.pdf}
    \caption{Probability density distributions of kinematic misalignment as defined by $\Delta$PA at $z=0$. In both panels the brightest 20\% in $L_{bol}$ (purple) compared with a mass matched control (green) is shown. The top panel shows the brightest in $L_{bol}$ at $z=0$ only, whereas the bottom panel shows those with the brightest peak $L_{bol}$ over the last 8 Gyrs. In each panel the Anderson-Darling statistic with a corresponding p-value is shown. We find no statistical difference between the active galaxies and the control for those selected at $z=0$ only, whereas we find that those which have been the most luminous over the last 8 Gyrs are distinctly more misaligned.}
    \label{fig:PAdist}
\end{figure}

We note that IllustrisTNG typically under-produces bright AGN ($L_{BH} > 10^{44}$ergs$^{-1}$) for $z \leq 1$ in contrast with observational constraints \citep[see][]{habouzit2019}. Given this and the uncertainty in estimating BH luminosity, we chose to select by percentile rather than cutting on absolute luminosity. Regardless, selecting only bright AGN in this way or choosing a higher percentile does not change our conclusions.

\section{Summary} \label{sec:conclusion}

In this paper, we study the relationship between BH luminosity, BH feedback and kinematic misalignment between stars and gas for galaxies in IllustrisTNG. We use mock observations of an IFS survey (MaNGA) built from galaxies in TNG100, to identify kinematic misalignment ($\Delta$PA; difference in PAs of stars and gas) at $z=0$. We split our mock IFS sample on mass to separate the impact of `quasar' and `kinetic' feedback modes. We follow the time evolution of BH luminosity and energy injection from BH feedback in leading up to misalignment (or counter-rotation) at $z=0$. We also compare the $z=0$ distributions of $\Delta$PA of the most luminous BHs in our sample against a control. Our conclusions are as follows.
\begin{enumerate}
    \item Low mass galaxies (quasar mode feedback) with misalignment (and counter-rotation) at $z=0$ typically have had boosted BH luminosity, BH growth and significantly more energy injected into the gas over the last 8 Gyr in comparison to aligned galaxies. Gas (all phases) is blown out due to the feedback, losing angular momentum and increasing metallicity towards $z=0$. This is seen for all populations split on $z=0$ galaxy type.
    
    \item The epoch of peak energy injection from the quasar mode feedback is different as a function of galaxy type at $z=0$ for misaligned galaxies. This can be explained by the relationship between energy injection from feedback and galaxy quenching (see Figure \ref{fig:diagram}). Misaligned quenched galaxies have typically experienced peak energy injection from the BH at earlier times which has since acted to suppress star formation, whereas misaligned star forming galaxies exhibit more recent energy injection. 

    \item Quenched high mass galaxies (both quasar and kinetic mode feedback) with misalignment (and counter-rotation) at $z=0$ typically have similar BH luminosity over the last 8 Gyr with respect to aligned galaxies and gas loss, if any, is less dramatic. Gas phase metallicity is also lower with respect to aligned galaxies. This suggests that the origin of misalignment in massive quenched galaxies is more likely due to accretion of pristine gas or loss of enriched gas.
    
    \item We find that the distributions of kinematic misalignment are statistically indistinct between the top 20\% in BH luminosity of low mass galaxies in our sample and a mass matched control at $z=0$. This matches observations \citep[see Figure 6 in][]{ilha2019}. Misalignment may initially occur at a similar time to the initial high accretion state (and hence peak BH luminosity), however misalignment can persist/correlate on much longer timescales. To test this we split by the top 20\% in maximum BH luminosity (for each galaxy) over the last 8 Gyrs in comparison with a control. We find that the most luminous AGN over the last 8 Gyrs are significantly more misaligned at $z=0$. Despite demonstrating the relationship between BH luminosity and misalignment in low mass galaxies, this result suggests you may not expect correlation with misalignment when considering BH activity at $z=0$ alone.
\end{enumerate}
% While this work clearly demonstrates the correlation between BH activity and kinematic misalignment, it makes no comment on causality. Further work is required to understand what triggers the initial high accretion mode and following period of feedback for both low and high mass galaxies.

% Kinematic misalignment can persist for far longer timescales than feedback and AGN luminosity. The exact relaxation timescales are of the order Gyrs but are dependent on the strength of the stellar torques in a galaxy \citep[e.g.][]{davis2016}. In addition, counter-rotation is a stable state and hence stars and gas can continue to remain decoupled. Further study is required to determine the origins of decoupled rotation in observations as a function of galaxies properties. Regardless, this work demonstrates the importance of feedback for low mass galaxies on large scale kinematics and points to an external origin of misalignment for massive ETGs.

\section*{Acknowledgements}
CD acknowledges support from the Science and Technology Funding Council (STFC) via an PhD studentship (grant number ST/N504427/1). The Flatiron Institute is supported by the Simons Foundation. We thank the IllustrisTNG team; while this work was conducted using the public data release, it would have not been possible without use of the private data for the purposes of other research works. 

%%%%%%%%%%%%%%%%%%%%%%%%%%%%%%%%%%%%%%%%%%%%%%%%%%

%%%%%%%%%%%%%%%%%%%% REFERENCES %%%%%%%%%%%%%%%%%%

% The best way to enter references is to use BibTeX:

\bibliographystyle{mnras}
\bibliography{biblio} 

%%%%%%%%%%%%%%%%%%%%%%%%%%%%%%%%%%%%%%%%%%%%%%%%%%

% Don't change these lines
\bsp	% typesetting comment
\label{lastpage}
\end{document}

% End of mnras_template.tex